\documentclass[12pt,unicode]{beamer}
\usepackage{luatexja}% 日本語したい
\renewcommand{\kanjifamilydefault}{\gtdefault}% 既定をゴシック体に
\usepackage{unicode-math}

\usepackage{graphicx}
\usepackage{ulem}

\usepackage{amsmath}
\usepackage{hyperref}

\input{style.tex}
\title{Transceternal Codegolf Technique}
\setbeamertemplate{footline}[frame number]
\setbeamertemplate{theorems}[numbered]
\setbeamerfont{itemize/enumerate body}{size=\fontsize{12pt}{12pt}}
\setbeamerfont{itemize/enumerate subbody}{size=\fontsize{10pt}{11,4pt}}
\begin{document}
\begin{frame}
   \titlepage{}
\end{frame}

\begin{frame}{おしながき}
   \tableofcontents
   % You might wish to add the option [pausesections]
\end{frame}
\section{概要}

\begin{frame}{ビフォーアフター}[fragile]
   当初のソースコード (1560バイト):
   before:
   \begin{verbatim}
   TODO 長すぎて入りきらない
   \end{verbatim}
\end{frame}
\begin{frame}{ビフォーアフター}[fragile]
   現在のソースコード (240バイト):
   \begin{verbatim}
012333435363738393a3b3c3d3e3f3g3h3i3j3k
3l3m3n3o3p3q3r3s3t3u3v3w3x3y3z323IK2Jzr
+,0-.W3X2XO3P3Q3R3S3T3U3V3W'(0)*VN3O#$0
%&UM3NYZ0!"TL3MAE2DB2C2yz/=2<:2;2zs[^2]
z\34_{2`z\|Υ2ΤB}2~2Σ2ΣΧΪ2ΩΨ2:Φ2Bίΰ0αβ}Ψ
Ϋή2έzά2Ψ2+FH2GBb>@2?:4[FFF
   \end{verbatim}
   (入りきらないので改行済)
\end{frame}

\begin{frame}{概要}[fragile]
Transceternal でソースコードを短くしたいときのチェックリスト。

ソースコードとは

\begin{verbatim}

catacat

\end{verbatim}

のような Transceternal 処理系に直接読み込めるものを指す

以降、ソースコード末尾は改行されていないものとする

   Transceternal 言語の基本的な仕様を把握していることを前提とする。
   言語原作者による解説 \url{https://esolangs.org/wiki/Transceternal}
   もしくは筆者のスライド \url{https://hiromi-mi.github.io/trans.pdf} を参照。

\end{frame}

\begin{frame}{空白文字をなくす}[fragile]
   \begin{description}
      \item[効果] 中規模以上のプログラムではソースコードを2/3 程度に減らせる。
      \item[実装の手間] 中
      \item[技巧性] 中
      \item[適用可能性] いつでも
   \end{description}

   Transceternal の単一トークンは 1文字 もしくは (スペース区切りの) 文字列 である

   この 1文字 には印字可能文字であればマルチバイト文字も含まれる。同一個数のトークンでも、ABC と表現するのと あ と表現するのでは
   全てのトークンを1文字で表記するとスペース区切りが不要になるので、ソースコードの空白が全て削除できる

   先の例では3バイト文字であるひらがなを用いたが、実際にはロシア文字などの 2バイト文字を使うとよい

   \begin{verbatim}

    # Cyriic, Arabic
    "".join([chr(x) for x in range(0x3a3, 0x052f)]) + "".join([chr(x) for x in range(0x61e, 0x70d)])
    \end{verbatim}

    Python ならこのように2バイト文字列を列挙できる (トークン数に応じ境界は調整すること)
\end{frame}

\begin{frame}{絶対アドレス指定用接点の共有}
   \begin{description}
      \item[効果] 高い。半減も可能?
      \item[実装の手間] 特殊な場合のみ実装するなら低、一般の場合は中から高
      \item[技巧性] 中
      \item[適用可能性] deserialize される接点が存在するならいつでも
   \end{description}
\end{frame}
\subsection{0ビットにする}
\begin{frame}
   \begin{description}
      \item[効果] 数bitから沢山まで、可能性はもろもろ
      \item[実装の手間] 高
      \item[技巧性] 高
      \item[適用可能性] いつでも
   \end{description}
\end{frame}

\begin{frame}{B1, admin, root 接点を他の接点と共有}
   \begin{description}
      \item[効果] 数バイト?
      \item[実装の手間] 高
      \item[技巧性] 高
      \item[適用可能性] 小規模プログラムのみ。B1 と比べadmin は難しい
   \end{description}
\end{frame}

\section{その他の要検討事項}
\begin{frame}

\end{frame}

\section{没となったアイデア}
\begin{frame}
   以下では没となったアイデアを取り上げる。
   場合によっては利用できるかもしれない?
\end{frame}

\begin{frame}{動的命令生成}
   問題点: allocate 命令をつかうために 4接点必要なこと、ソースコードが読みにくくなること

   allocate おコストが高いので利用可能な場面に思い至らない状況
\end{frame}

\begin{frame}{ブックマーク (仮)}
   deserialize 対象について、接点が 1 と判定される基準は 0-pointer が B0 でないこと。
   B1 でなくてもよい。

   "B0 以外" の接点を今までは B1 などと定めていたが、そこに情報を埋め込みたい

   有益な接点にアクセスするための絶対アドレスがdeserialize 対象の絶対アドレス + 最初の1 までのアドレス数 + 0 で表現できる。深い階層にある変数を簡潔に表せることが期待できる
\end{frame}

\begin{frame}{ブックマーク (仮)}
   if 扱いになる条件は また、 B0 でもB1 でもない接点を指すとき であった

   その接点を新しく作らずに既存接点 (root) にすることで if に必要な接点が1つ省略できるのは紹介した通りだが、そこに情報を埋め込みたい

   プログラム中実行中に if 対象の接点は絶対アドレス  `0100` のたった4バイトでアクセスできる。
   深い階層にある変数をif 中の左辺や右辺に使うとき、その変数の接点へのポインタを `0100` に置いておくと 必要な絶対アドレスの abstruct 表現が削減できる。

   今回は変数の個数が少なく、全ての変数を B0, B1 直下に配置したので、ブックマークせずとも元々の位置で既に4バイト以下でアクセスできた。この方針では有益にならなかった。
\end{frame}


\end{document}
